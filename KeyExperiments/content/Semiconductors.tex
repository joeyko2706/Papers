\chapter{Semiconductor Detectors}\label{Semiconductors}

The earliest studies of semiconductor detectors date back to 1833 when Faraday discovered the temperature dependence of the conductivity of silver sulphide.
Nearly a hundred years later, Wilson described a band theory of solids in 1931, in which quantum mechanics explains the characteristics of semiconductors.
The electrical conductivity can be described by the valence band (VB) which consists of positively charged quasiparticles that are called holes $h^+$ and the conduction band (CB) which is filled with electrons \cite{KolanoskiWermes}.
The band theory of solids distinguishes three types of conductive solids.
Insulators with a band gap of $E_G\approx \SI{9}{\eV}$, semiconductors with $E_G\approx \SI{1}{\eV}$ and conductors with no gap.
The properties of semiconductors can be changed by inserting impurities into the crystal structure.
If an atom with five electrons, called a donor, is inserted into an element with four electrons, an excessive conduction electron appears.
If an atom with three electrons, called an acceptor, is inserted into an element with three electrons, an excessive hole appears.
Semiconductors doped with donors are called n-doped while those doped with acceptors are called p-doped.
When both types of doped semiconductors are combined they form a diode.
Drifting and combining of $e^-/h^+$ forms an electrical field at the junction of the semiconductor which is called the depletion zone.
If an external positive voltage at the p side relative to the n side is applied, this electrical field is weakened and current can flow which is called forward bias.
In reverse bias, the electrical field is strengthened and the depletion zone increases. 
This configuration is used in detectors to measure ionizing particles.
\begin{figure}[H]
    \centering
    \includegraphics[width=0.69\textwidth]{figs/MOS.png}
    \caption{Schematic illustration of a MOSFET and a CMOS semiconductor \cite{KolanoskiWermes}.}
    \label{fig:MOStransition}
\end{figure}
The most used semiconductor detectors are complementary metal-oxide-semiconductors (CMOS) consisting of a part with a p-doped substrate and an n-doped drain terminal (NMOS) and a part with the reverse (PMOS).
Figure \ref{fig:MOStransition} shows a CMOS and a widely in field-effect transistor used semiconductor (MOSFET).

The simplest design of detectors is a pn area diode consisting of a $\SI{300}{\micro\meter}$ thick p and n-doped area.
An additional guard ring can reduce the electrical noise.
Dividing the area into strips or pixels ($\text{length}<\SI{100}{\micro\meter}$) yields one or even two spatial coordinates but increases the readout difficulty.
There are two different types to read out the information from the collection diodes.
Hybrid pixel sensors have the readout electronics on a different chip which is a laborious assembly and resides in a large material thickness, the sensor, however, can be separately optimized to the readout components.
Monolithic pixel sensors reduce their material by an entire order of magnitude but not all production lines are suited to produce such sensors.
As pixel detectors are used to reconstruct traces for example in the ATLAS Experiment due to their good resolution with an uncertainty of $\sigma_x=\sfrac{\text{pitch}}{\sqrt{12}}$ \cite{Tom}, they are placed closest to the collision vertex where a lot of charged particles pass through them causing damage to the detector substrate.
Such radiation damage can cause a change in effective doping concentration leading to deactivated donor or acceptor atoms.
They can furthermore cause trapping where $e^-/h^+$ are trapped in defects of the crystal structure and are released later.
These defects can form energy levels that excite $e^-/h^+$ easier and cause a flow in current.
Radiation damage can be reduced by cooling \cite{Tom}.
\begin{wrapfigure}{l}{7.5cm}
    \centering
    \includegraphics[width=0.5\textwidth]{figs/MightyPix.png}
    \caption{Schematic illustration of a MightyPix \cite{MightyPix}.}
    \label{fig:MightyPix}
\end{wrapfigure}
A track is reconstructed by forming groups of measurement points, called tracklets which require a good spatial and time resolution of the pixel detectors.
In each of the tracklets every possible combination is built but only those that point to the interacting point are accepted, forming a track when a series of tracklets match.
An application of pixel detectors is given in the example of MightyPix, a design sketch of possible pixel detectors used in the second upgrade of the LHC that is shown in figure \ref{fig:MightyPix}.
Their requirements for the upgrade include a time resolution of $<\SI{3}{\nano\second}$, and a pixel size of $~\SI{50}{\micro\meter}\times\SI{150}{\micro\meter}$.
Semiconductor detectors in the form of pixel detectors provide good time and space resolution which is why particle detectors are indispensable without them.
However, the implementation of pixel detectors brings new challenges like the power distribution or the cooling. %of the detectors.
